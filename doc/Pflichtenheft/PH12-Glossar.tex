\section{Glossar}

Zur Sicherstellung einer einheitlichen Terminologie sind hier nochmal alle
Kernbegriffe definiert.

\begin{description}
  \item[RSS-Feed]
    Ein RSS-Feed stellt Nachrichten auf einer Webseite in einer Art
    XML-Liste dar. Neuerungen sind durch Zeitstempel einzelner
    Listeneintr�ge (Feed) im XML-Dokument erkennbar.
  \item[Feed]
    Jeder einzelne \textit{Listeneintrag} stellt eine neue \textit{News}
    dar. Dies ist ein neuer Feed.
  \item[Feed-Liste]
    Das ist die prim�re Anwendung der App und ein Startpunkt. Alle Feeds
    werden hier aufgelistet, sofern der Benutzer diese nicht gel�scht hat.
  \item[Benutzer]
    Der Benutzer oder die Benutzerin ist die Person, die sich die App installiert
    hat und nutzt.
  \item[Einstellungen, Optionen]
    Die Options- oder Einstellungs-Activity ist f�r den Benutzter da, damit er/sie
    in erster Linie den Nachrichten-Dienst konfigurieren und einstellen kann.
  \item[Nachricht]
    Um dem Benutzer auf einen neuen Feed aufmerksam zu machen, wird das Android
    Notify System genutzt. Die Nachricht zu einem neuen Feed wird als Meldung
    (Benachrichtigung) gezeigt.
  \item[Dienst, Service, Nachrichten-Dienst]
    Ein Dienst sorgt daf�r, dass man die App nicht ge�ffnet haben muss,
    um via Nachricht auf einen neuen Feed hingewiesen zu werden. Der Intervall,
    wie oft der Dienst nach neuen Feeds schauen soll, ist �ber die Optionen
    der App einstellbar.
  \item[URL]
    Ein RSS-Feed wird im Internet wie eine Webseite zur Verf�gung gestellt.
    Durch die Angabe einer URL ist diese Webseite technisch lesbar und
    Links, Datum, Titel und Inhalt sind verarbeitbar.
  
\end{description}
